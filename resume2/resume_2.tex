%-------------------------
% Based off of: https://github.com/sb2nov/resume
%------------------------

\documentclass[letterpaper,12pt]{article}

\usepackage{latexsym}
\usepackage[empty]{fullpage}
\usepackage{titlesec}
\usepackage{marvosym}
\usepackage[usenames,dvipsnames]{color}
\usepackage{verbatim}
\usepackage{enumitem}
\usepackage[hidelinks]{hyperref}
\usepackage{fancyhdr}
\usepackage[english]{babel}
\usepackage{tabularx}
\usepackage{fontawesome5}
\usepackage{multicol}
\input{glyphtounicode}


%----------FONT OPTIONS----------
% sans-serif
% \usepackage[sfdefault]{FiraSans}
% \usepackage[sfdefault]{roboto}
% \usepackage[sfdefault]{noto-sans}
% \usepackage[default]{sourcesanspro}

% serif
% \usepackage{CormorantGaramond}
% \usepackage{charter}


\pagestyle{fancy}
\fancyhf{} % clear all header and footer fields
\fancyfoot{}
\renewcommand{\headrulewidth}{0pt}
\renewcommand{\footrulewidth}{0pt}

% Adjust margins
\addtolength{\oddsidemargin}{-0.5in}
\addtolength{\evensidemargin}{-0.5in}
\addtolength{\textwidth}{1in}
\addtolength{\topmargin}{-.5in}
\addtolength{\textheight}{1.0in}

\urlstyle{same}

\raggedbottom
\raggedright
\setlength{\tabcolsep}{0in}

% Sections formatting
\titleformat{\section}{
  \vspace{-4pt}\scshape\raggedright\large
}{}{0em}{}[\color{black}\titlerule \vspace{-5pt}]

% Ensure that generate pdf is machine readable/ATS parsable
\pdfgentounicode=1

%-------------------------
% Custom commands
\newcommand{\resumeItem}[1]{
  \item\small{
    {#1 \vspace{-2pt}}
  }
}

\newcommand{\resumeSubheading}[4]{
  \vspace{-2pt}\item
    \begin{tabular*}{0.97\textwidth}[t]{l@{\extracolsep{\fill}}r}
      \textbf{#1} & #2 \\
      \textit{\small#3} & \textit{\small #4} \\
    \end{tabular*}\vspace{-7pt}
}

\newcommand{\resumeSubheadingNoSkill}[2]{
    \item
    % \begin{tabular*}{0.97\textwidth}{l@{\extracolsep{\fill}}r}
    \begin{tabular*}{0.97\textwidth}[t]{l@{\extracolsep{\fill}}r}
      \textbf{#1} & \textit{\small #2} \\
    \end{tabular*}\vspace{-7pt}
}

\newcommand{\resumeProjectHeading}[2]{
    \item
    \begin{tabular*}{0.97\textwidth}{l@{\extracolsep{\fill}}r}
      \small#1 & #2 \\
    \end{tabular*}\vspace{-7pt}
}

\newcommand{\resumeSubItem}[1]{\resumeItem{#1}\vspace{-4pt}}
\newcommand{\resumeCourseItem}[4]{\resumeItem{#1}\vspace{-4pt}}

\renewcommand\labelitemii{$\vcenter{\hbox{\tiny$\bullet$}}$}

\newcommand{\resumeSubHeadingListStart}{\begin{itemize}[leftmargin=0.05in, label={}]}
\newcommand{\resumeSubHeadingListEnd}{\end{itemize}}
\newcommand{\resumeItemListStart}{\begin{itemize}}
\newcommand{\resumeItemListEnd}{\end{itemize}\vspace{-5pt}}

%-------------------------------------------
%%%%%%  RESUME STARTS HERE  %%%%%%%%%%%%%%%%%%%%%%%%%%%%


\begin{document}


\begin{center}
    \textbf{\Huge \scshape James Young} \\ \vspace{2pt}
    \small  \faPhone\ {+852 95731718} $|$ 
    \href{mailto:jyyoungaa@connect.ust.hk}{\faEnvelope { jyyoungaa@connect.ust.hk}} $|$ 
    \href{https://github.com/jamesyoung-15} {\faGithub\ {jamesyoung-15}} $|$
    \href{https://linkedin.com/in/jamesyyoung}{\faLinkedin { linkedin.com/in/jamesyyoung}}
\end{center}


%-----------EDUCATION-----------
\section{Education}
  \resumeSubHeadingListStart
    \resumeSubheading
      {The Hong Kong University of Science and Technology}{Hong Kong}
      {BEng in Electronic Engineering - Minor in Information Technology}{Sept. 2020 -- June 2024}
  \resumeSubHeadingListEnd

\section{Relevant Coursework}
\vspace{-12pt}
    \begin{multicols}{2}
        \begin{itemize}[itemsep=-1pt, parsep=3pt]
        \small
            % \item Intro. to Computer Science
            \item Programming with C++
            \item Object-Oriented Programming and Data Structures
            \item Intro. to Computer Organization and Design
            \item Computer Communication Networks
            \item Machine Learning and Information Processing for Robotics
            \item Introduction to Embedded Systems
        \end{itemize}
    \end{multicols}

%-----------SKILLS-----------
\vspace{-20pt}
\section{Skills}
 \begin{itemize}[leftmargin=0.15in, label={}]
    \small{\item{
     \textbf{Programming Languages}{: C++, Python} \\
     \textbf{Tools/Platforms}{: Git, Docker, Linux, Ansible, Jenkins} \\
     \textbf{Hardware}{: STM32, Raspberry PI, ESP32, Arduino} \\
     \textbf{Protocols}{: UART, SPI, I2C, ADC} \\
    }}
 \end{itemize}

%-----------EXPERIENCE-----------
\section{Work Experience}
  \resumeSubHeadingListStart
    \resumeSubheadingNoSkill
        {\href{https://github.com/jamesyoung-15/FallDetection}{{Software Developer Intern $|$ Intelligent Design Technology}}}{December 2023 -- Feburary 2024}
        % {Python}{(Full Time)}
        % \vspace{1pt}
        \resumeItemListStart 
          \resumeItem{\normalsize{Developed a prototype for real-time human fall detection for a Raspberry PI based robot.}}
          \resumeItem{\normalsize{Fall detection uses Tensorflow and Movenet for pose estimation and heuristics for determining fall.}}
          % \resumeItem{\normalsize{Created an Arduino library for communicating to multiple ESP32 with ESP-Now protocol.}}
        \resumeItemListEnd

    \resumeSubheadingNoSkill
      {\href{https://github.com/jamesyoung-15/ESP32Cam-MeterReading}{{Electronic Engineering Intern $|$ Kolour Think Tank}}}{August 2023}
      % {\href{https://github.com/jamesyoung-15/ESP32Cam-MeterReading}{{Electronic Engineering Intern $|$ Kolour Think Tank}\small{\faExternalLink*}}}{August 2023}
      % {ESP32, AWS}{(Full Time)}
      % \vspace{1pt}
      \resumeItemListStart 
        \resumeItem{\normalsize{Developed a digital utility meter reader that takes images of a utility meter with an ESP32-CAM, sends the image to AWS S3, reads the meter reading with AWS Rekognition, and stores the data in AWS DynamoDB.}}
        % \resumeItem{\normalsize{Created an Arduino library for communicating to multiple ESP32 with ESP-Now protocol.}}
      \resumeItemListEnd

    \resumeSubheadingNoSkill
      {\href{https://github.com/jamesyoung-15/Internship-GraphiteVentureLimited}{{IoT Intern $|$ Graphite Venture Limited}}}{December 2022 -- May 2023}
      % {C++, ESP32, Arduino}{(Full and Part Time)}
      % \vspace{1pt}
      \resumeItemListStart 
        \resumeItem{\normalsize{Developed Arduino libraries for reading water sensor data with ESP32 and sending sensor data to AWS IoT Core through MQTT with a SIM7600G module.}}
        % \resumeItem{\normalsize{Created an Arduino library for communicating to multiple ESP32 with ESP-Now protocol.}}
      \resumeItemListEnd


    

  \resumeSubHeadingListEnd

%-----------Personal PROJECTS-----------
\section{Projects}
    \resumeSubHeadingListStart
      
    \resumeProjectHeading
    {\textbf{\href{https://github.com/jamesyoung-15/Mini-Robot-Cleaner}{ {Mini Robot Cleaner} \small{\faExternalLink*}}}    \emph{}}{}
    \resumeItemListStart
      \resumeItem{\normalsize{Created a robot car with a STM32 board that can be wirelessly controlled through UDP or can roam autonomously}} 
      \resumeItem{\normalsize{Integrated the bubble rebound algorithm for avoiding obstacles in free roam mode using 3 ultrasonic sensors}}
      \resumeItem{\normalsize{Used Python for socket programming and PyQT5 to create GUI to control robot wirelessly}}
    \resumeItemListEnd

    \resumeProjectHeading
    {\textbf{\href{https://airqualitydashboard.jyylab.com/}{ {IoT Air Quality Monitoring} \small{\faExternalLink*}}}    \emph{}}{}
    \resumeItemListStart
      \resumeItem{\normalsize{An IoT air quality monitoring system that uses STM32 to measure data from SGP30 and AM2320 sensors and sends the data to ESP32 to display on web browser.}} 
      \resumeItem{\normalsize{Uses I2C for obtaining sensor data, SPI to display information on an LCD, and UART to send data to ESP32.}}
    \resumeItemListEnd



      % \resumeProjectHeading
      % {\textbf{\href{https://github.com/jamesyoung-15/Tic-Tac-Toe-Minimax}{ {Tic-Tac-Toe with Minimax} \faExternalLink*}} $|$ \emph{}}{}
      % \resumeItemListStart
      %   \resumeItem{\normalsize{A terminal tic-tac-toe game with option to either play against AI or another person.}}
      %   \resumeItem{\normalsize{AI uses minimax algorithm to determine the best move each turn. Used tree data-structure for storing different board states in each turn.}}
      % \resumeItemListEnd
      



    \resumeSubHeadingListEnd








%------Extra Curricular-------
\section{Extracurricular Activities}
    \begin{itemize}[itemsep=-2pt, parsep=5pt] 
        \item  {HKUST Football Team \hfill Jan. 2021 - June 2024}
    \end{itemize}



\end{document}
