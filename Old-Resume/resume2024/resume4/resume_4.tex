%-------------------------
% Based off of: https://github.com/sb2nov/resume
%------------------------

\documentclass[letterpaper,12pt]{article}

\usepackage{latexsym}
\usepackage[empty]{fullpage}
\usepackage{titlesec}
\usepackage{marvosym}
\usepackage[usenames,dvipsnames]{color}
\usepackage{verbatim}
\usepackage{enumitem}
\usepackage[hidelinks]{hyperref}
\usepackage{fancyhdr}
\usepackage[english]{babel}
\usepackage{tabularx}
\usepackage{fontawesome5}
\usepackage{multicol}
\input{glyphtounicode}


%----------FONT OPTIONS----------
% sans-serif
% \usepackage[sfdefault]{FiraSans}
% \usepackage[sfdefault]{roboto}
% \usepackage[sfdefault]{noto-sans}
% \usepackage[default]{sourcesanspro}

% serif
% \usepackage{CormorantGaramond}
% \usepackage{charter}


\pagestyle{fancy}
\fancyhf{} % clear all header and footer fields
\fancyfoot{}
\renewcommand{\headrulewidth}{0pt}
\renewcommand{\footrulewidth}{0pt}

% Adjust margins
\addtolength{\oddsidemargin}{-0.5in}
\addtolength{\evensidemargin}{-0.5in}
\addtolength{\textwidth}{1in}
\addtolength{\topmargin}{-.5in}
\addtolength{\textheight}{1.0in}

\urlstyle{same}

\raggedbottom
\raggedright
\setlength{\tabcolsep}{0in}

% Sections formatting
\titleformat{\section}{
  \vspace{-4pt}\scshape\raggedright\large
}{}{0em}{}[\color{black}\titlerule \vspace{-5pt}]

% Ensure that generate pdf is machine readable/ATS parsable
\pdfgentounicode=1

%-------------------------
% Custom commands
\newcommand{\resumeItem}[1]{
  \item\small{
    {#1 \vspace{-2pt}}
  }
}

\newcommand{\resumeSubheading}[4]{
  \vspace{-2pt}\item
    \begin{tabular*}{0.97\textwidth}[t]{l@{\extracolsep{\fill}}r}
      \textbf{#1} & #2 \\
      \textit{\small#3} & \textit{\small #4} \\
    \end{tabular*}\vspace{-7pt}
}

\newcommand{\resumeSubheadingNoSkill}[2]{
    \item
    % \begin{tabular*}{0.97\textwidth}{l@{\extracolsep{\fill}}r}
    \begin{tabular*}{0.97\textwidth}[t]{l@{\extracolsep{\fill}}r}
      \textbf{#1} & \textit{\small #2} \\
    \end{tabular*}\vspace{-7pt}
}

\newcommand{\resumeProjectHeading}[2]{
    \item
    \begin{tabular*}{0.97\textwidth}{l@{\extracolsep{\fill}}r}
      \small#1 & #2 \\
    \end{tabular*}\vspace{-7pt}
}

\newcommand{\resumeSubItem}[1]{\resumeItem{#1}\vspace{-4pt}}
\newcommand{\resumeCourseItem}[4]{\resumeItem{#1}\vspace{-4pt}}

\renewcommand\labelitemii{$\vcenter{\hbox{\tiny$\bullet$}}$}

\newcommand{\resumeSubHeadingListStart}{\begin{itemize}[leftmargin=0.05in, label={}]}
\newcommand{\resumeSubHeadingListEnd}{\end{itemize}}
\newcommand{\resumeItemListStart}{\begin{itemize}}
\newcommand{\resumeItemListEnd}{\end{itemize}\vspace{-5pt}}

%-------------------------------------------
%%%%%%  RESUME STARTS HERE  %%%%%%%%%%%%%%%%%%%%%%%%%%%%


\begin{document}


\begin{center}
    \textbf{\Huge \scshape James Young} \\ \vspace{2pt}
    \href{mailto:jyyoung@bu.edu}{\faEnvelope { jyyoung@bu.edu}} $|$ 
    \href{https://github.com/jamesyoung-15} {\faGithub\ {jamesyoung-15}} $|$
    \href{https://linkedin.com/in/jamesyyoung}{\faLinkedin { linkedin.com/in/jamesyyoung}}
\end{center}


%-----------EDUCATION-----------
\section{Education}
  \resumeSubHeadingListStart
    \resumeSubheading
      {The Hong Kong University of Science and Technology}{Hong Kong}
      {BEng in Electronic Engineering - Minor in Information Technology}{Sept. 2020 -- June 2024}
  \resumeSubHeadingListEnd
  \resumeSubHeadingListStart
    \resumeSubheading
      {Boston University}{United States}
      {Master in Computer Science}{Sept. 2024 -- Present}
  \resumeSubHeadingListEnd
\vspace{-5pt}

%-----------Certs-----------
\section{Certifications}
\vspace{-12pt}
% \vspace{1pt}
    \begin{multicols}{2}
        \begin{itemize}[itemsep=-1pt, parsep=3pt, leftmargin=0.21in]
        \small
            % \item CompTIA A+ Certification
            % \item Certified Kubernetes Administrator (CKA)
            \item AWS Certified Solutions Architect (SAA-C03)
            \item Red Hat Certified System Administrator (RHCSA)
            \item HashiCorp Certified Terraform Associate (003)
            % \item Red Hat Certified Engineer (RHCE)
        \end{itemize}
    \end{multicols}

%-----------SKILLS-----------
% \vspace{-25pt}
\section{Skills}
 \begin{itemize}[leftmargin=0.15in, label={}]
    \small{\item{
     \textbf{Programming Languages}{: Python, Bash} \\
     \textbf{Tools}{: AWS, Terraform, Docker, Ansible, Git, Github Actions} \\
     \textbf{Platforms}{: Linux, Windows, MacOS, Proxmox, VMWare} \\ 
     \textbf{Others}{: Networking Concepts (VLANs, VPN, DNS)}, PC Hardware \\
    }}
 \end{itemize}

%-----------EXPERIENCE-----------
\section{Work Experience}
  \resumeSubHeadingListStart

    \resumeSubheading
      {{{Intelligent Design Technology}}}{Hong Kong}
      {Software Developer Intern}{December 2023 -- Feburary 2024}
      \vspace{1pt}
      \resumeItemListStart 
      \resumeItem{\normalsize{Developed a prototype for real-time human fall detection for a Raspberry PI based robot using camera inputs with Python.}}
      \resumeItem{\normalsize{Fall detection uses Tensorflow and Movenet for pose estimation and heuristics for determining fall.}}
      \resumeItemListEnd

    % \resumeSubheadingNoSkill
    %     {{{Electronic Engineering Intern $|$ Kolour Think Tank}}}{July 2023 -- August 2023}
    %     % {\href{https://github.com/jamesyoung-15/ESP32Cam-MeterReading}{{Electronic Engineering Intern $|$ Kolour Think Tank}\small{\faExternalLink*}}}{August 2023}
    %     % {ESP32, AWS}{(Full Time)}
    %     % \vspace{1pt}
    %     \resumeItemListStart 
    %       \resumeItem{\normalsize{Helped create an IoT digital utility meter reader using AWS services and an ESP32-Cam.}}
    %       \resumeItem{\normalsize{Programmed ESP32-Cam to take images of a utility meter and send the images to S3. Used Rekognition OCR on images to obtain meter reading and stored numerical data into DynamoDB.}}
    %     \resumeItemListEnd

    \resumeSubheading
      {{{Graphite Venture Limited}}}{Hong Kong}
      {IoT Intern}{December 2022 -- May 2023}
      % \vspace{1pt}
      \resumeItemListStart 
        \resumeItem{\normalsize{Created Arduino libraries in C++ for reading water sensor data with ESP32 and sending sensor data to AWS IoT Core with MQTT.}}
        \resumeItem{\normalsize{Achieved 40\% decrease in power consumption by implementing light sleep intervals in ESP32.}}
      \resumeItemListEnd



  \resumeSubHeadingListEnd

%-----------Personal PROJECTS-----------
\section{Projects}
    \resumeSubHeadingListStart

    \resumeProjectHeading
        {\textbf{\href{https://github.com/jamesyoung-15/homeserver}{ {Home Server}}} \emph{}}{}
        \resumeItemListStart
          \resumeItem{\normalsize{Built homelab running Proxmox using Linux containers and VMs for self-hosting services such as file and media servers, Github self-hosted runners, etc.}}
          \resumeItem{\normalsize{Applied tools such as Grafana for monitoring, Terraform and Ansible for deploying and provisioning resources, etc.}}
        \resumeItemListEnd

    \resumeProjectHeading
      {\textbf{\href{https://github.com/jamesyoung-15/aws-serverless-face-blurring}{ {Serverless Face Blurring}}}    \emph{}}{}
      \resumeItemListStart
        \resumeItem{\normalsize{An event-driven serverless application that allows users to upload images and blurs faces on the images.}} 
        \resumeItem{\normalsize{Image processing uses Lambda and Rekognition, backend uses S3 and DynamoDB with SQS in-between to decouple the application.}}
        \resumeItem{\normalsize{Used Terraform for deployment and implemented basic CI/CD pipeline using Github Actions.}}
      \resumeItemListEnd




      % \resumeProjectHeading
      % {\textbf{\href{https://airqualitydashboard.jyylab.com/}{ {Air Quality Monitoring Dashboard}}}    \emph{}}{}
      % \resumeItemListStart
      %   \resumeItem{\normalsize{A fullstack project that stores and displays my home's air quality sensor data using AWS services.}} 
      %   \resumeItem{\normalsize{Data is stored and retrieved on DynamoDB using REST API with API Gateway and Lambda. Application deployed on an EC2 instance.}}
      % \resumeItemListEnd




    \resumeSubHeadingListEnd


% Replace with Certs in the future



% ------Extra Curricular-------
% \section{Extracurricular Activities}
%     \begin{itemize}[itemsep=-2pt, parsep=5pt] 
%         \item  {HKUST Football Team \hfill Jan. 2021 - June 2024}
%     \end{itemize}



\end{document}
