%-------------------------
% Based off of: https://github.com/sb2nov/resume
%------------------------

\documentclass[letterpaper,11pt]{article}

\usepackage{latexsym}
\usepackage[empty]{fullpage}
\usepackage{titlesec}
\usepackage{marvosym}
\usepackage[usenames,dvipsnames]{color}
\usepackage{verbatim}
\usepackage{enumitem}
\usepackage[hidelinks]{hyperref}
\usepackage{fancyhdr}
\usepackage[english]{babel}
\usepackage{tabularx}
\usepackage{fontawesome5}
\usepackage{multicol}
\input{glyphtounicode}


%----------FONT OPTIONS----------
% sans-serif
% \usepackage[sfdefault]{FiraSans}
% \usepackage[sfdefault]{roboto}
% \usepackage[sfdefault]{noto-sans}
% \usepackage[default]{sourcesanspro}

% serif
% \usepackage{CormorantGaramond}
% \usepackage{charter}


\pagestyle{fancy}
\fancyhf{} % clear all header and footer fields
\fancyfoot{}
\renewcommand{\headrulewidth}{0pt}
\renewcommand{\footrulewidth}{0pt}

% Adjust margins
\addtolength{\oddsidemargin}{-0.5in}
\addtolength{\evensidemargin}{-0.5in}
\addtolength{\textwidth}{1in}
\addtolength{\topmargin}{-.5in}
\addtolength{\textheight}{1.0in}

\urlstyle{same}

\raggedbottom
\raggedright
\setlength{\tabcolsep}{0in}

% Sections formatting
\titleformat{\section}{
  \vspace{-4pt}\scshape\raggedright\large
}{}{0em}{}[\color{black}\titlerule \vspace{-5pt}]

% Ensure that generate pdf is machine readable/ATS parsable
\pdfgentounicode=1

%-------------------------
% Custom commands
\newcommand{\resumeItem}[1]{
  \item\small{
    {#1 \vspace{-2pt}}
  }
}

\newcommand{\resumeSubheading}[4]{
  \vspace{-1pt}\item
    \begin{tabular*}{0.98\textwidth}[t]{l@{\extracolsep{\fill}}r}
      \textbf{#1} & #2 \\
      \textit{\small#3} & \textit{\small #4} \\
    \end{tabular*}\vspace{-7pt}
}

\newcommand{\resumeSubheadingNoSkill}[2]{
    \item
    % \begin{tabular*}{0.97\textwidth}{l@{\extracolsep{\fill}}r}
    \begin{tabular*}{0.97\textwidth}[t]{l@{\extracolsep{\fill}}r}
      \textbf{#1} & \textit{\small #2} \\
    \end{tabular*}\vspace{-7pt}
}

\newcommand{\resumeProjectHeading}[2]{
    \item
    \begin{tabular*}{0.97\textwidth}{l@{\extracolsep{\fill}}r}
      \small#1 & #2 \\
    \end{tabular*}\vspace{-7pt}
}

\newcommand{\resumeSubItem}[1]{\resumeItem{#1}\vspace{-4pt}}
\newcommand{\resumeCourseItem}[4]{\resumeItem{#1}\vspace{-4pt}}

\renewcommand\labelitemii{$\vcenter{\hbox{\tiny$\bullet$}}$}

\newcommand{\resumeSubHeadingListStart}{\begin{itemize}[leftmargin=0.05in, label={}]}
\newcommand{\resumeSubHeadingListEnd}{\end{itemize}}
\newcommand{\resumeItemListStart}{\begin{itemize}}
\newcommand{\resumeItemListEnd}{\end{itemize}\vspace{-5pt}}

%-------------------------------------------
%%%%%%  RESUME STARTS HERE  %%%%%%%%%%%%%%%%%%%%%%%%%%%%


\begin{document}


\begin{center}
    \textbf{\Huge \scshape James Young} \\ \vspace{3pt}
    % \small  \faPhone\ {+1 (857) 270-8026} $|$ 
    \href{mailto:jyyoung@bu.edu}{\faEnvelope { jyyoung@bu.edu}} $|$ 
    \href{https://github.com/jamesyoung-15} {\faGithub\ {jamesyoung-15}} $|$
    \href{https://linkedin.com/in/jamesyyoung}{\faLinkedin { linkedin.com/in/jamesyyoung}}
\end{center}


%-----------EDUCATION-----------
\section{Education}
  \resumeSubHeadingListStart
    \resumeSubheading
      {The Hong Kong University of Science and Technology}{Hong Kong}
      {BEng in Electronic Engineering - Minor in Information Technology}{Sept. 2020 -- June 2024}
  \resumeSubHeadingListEnd
  \vspace{-7pt}
  \resumeSubHeadingListStart
    \resumeSubheading
      {Boston University}{United States}
      {Master in Computer Science}{Sept. 2024 -- Present}
  \resumeSubHeadingListEnd
\vspace{-10pt}


%-----------SKILLS-----------
% \vspace{-15pt}
\section{Skills}
 \begin{itemize}[leftmargin=0.05in, label={}]
    \small{\item{
     \textbf{Programming Languages}{: Python, Javascript, C++} \\
     \textbf{Tools/Platforms}{: AWS, Linux, Git, Terraform, Docker, Github Actions} \\ 
    %  \textbf{Cloud}{: AWS (Lambda, EC2, S3, DynamoDB)} \\
     \textbf{Others}{: Networking Concepts (VLANs, VPN, DNS, etc.)} \\
    }}
 \end{itemize}

%-----------EXPERIENCE-----------
\section{Work Experience}
  \resumeSubHeadingListStart

    \resumeSubheading
        {{{Intelligent Design Technology}}}{Hong Kong}
        {Software Developer Intern}{December 2023 -- Feburary 2024}
        \vspace{1pt}
        \resumeItemListStart 
          \resumeItem{\normalsize{Developed a prototype for real-time human fall detection for a Raspberry PI based robot in Python.}}
          \resumeItem{\normalsize{Utilized Tensorflow's Movenet for pose estimation combined with heuristics for determining fall.}}
        \resumeItemListEnd

    \resumeSubheading
        {{{Kolour Think Tank}}}{Hong Kong}
        {Electronic Engineering Intern}{July 2023 -- August 2023}
        \vspace{1pt}
        \resumeItemListStart 
          \resumeItem{\normalsize{Developed a low-cost IoT serverless digital utility meter reader using ESP32-Cam and AWS services.}}
          \resumeItem{\normalsize{Utilized S3 for storing images of utility meter, Lambda and Rekognition to extract image data, and DynamoDB for data storage.}}
        \resumeItemListEnd

    \resumeSubheading
      {{{Graphite Venture Limited}}}{Hong Kong}
      {IoT Intern}{December 2022 -- May 2023}
      % \vspace{1pt}
      \resumeItemListStart 
        \resumeItem{\normalsize{Created Arduino libraries in C++ for reading water sensor data with ESP32 and sending sensor data to AWS IoT Core.}}
        \resumeItem{\normalsize{Achieved 40\% decrease in power consumption by implementing light sleep intervals in ESP32.}}
      \resumeItemListEnd


% Worked on a low-cost IoT serverless digital utility meter reader with ESP32-Cam and integrated with API Gateway, S3, Rekognition, Lambda, and DynamoDB.}}
  \resumeSubHeadingListEnd

%-----------Personal PROJECTS-----------
\section{Projects}
    \resumeSubHeadingListStart

    \resumeProjectHeading
      {\textbf{\href{https://github.com/jamesyoung-15/aws-serverless-face-blurring}{ {Serverless Face Blurring}}}    \emph{}}{}
      \resumeItemListStart
        \resumeItem{\normalsize{An event-driven serverless application that allows users to upload images and blurs faces on the images.}} 
        \resumeItem{\normalsize{Image processing uses Lambda and Rekognition, backend uses S3 for image storage and DynamoDB for job ID tracking with SQS to decouple application.}}
        \resumeItem{\normalsize{Front-end uses vanilla HTML, CSS, Javascript and compresses image client-side.}} 
        \resumeItem{\normalsize{Used Terraform for deployment and implemented CI/CD pipeline using Github Actions.}}
      \resumeItemListEnd

    \resumeProjectHeading
        {\textbf{\href{https://github.com/jamesyoung-15/homeserver}{ {Home Server}}} \emph{}}{}
        \resumeItemListStart
          \resumeItem{\normalsize{Built homelab running Proxmox using Linux containers and VMs for self-hosting services such as file and media servers, Github self-hosted runners, etc.}}
          \resumeItem{\normalsize{Applied tools such as Grafana for monitoring, Terraform and Ansible for deploying and provisioning resources, etc.}}
        \resumeItemListEnd



    \resumeSubHeadingListEnd

%-----------Certs-----------
\section{Certifications}
% \vspace{-15pt} % if using multicols
\vspace{2pt}
    % \begin{multicols}{2}
        \begin{itemize}[itemsep=-1pt, parsep=3pt, leftmargin=0.22in]
        \small
            \item AWS Certified Solutions Architect (SAA-C03)
            \item Red Hat Certified System Administrator (RHCSA)
            \item HashiCorp Certified Terraform Associate (003)
        \end{itemize}
    % \end{multicols}

\end{document}
