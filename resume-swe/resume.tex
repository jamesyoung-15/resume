\documentclass[letterpaper,11pt]{article}

\usepackage{latexsym}
\usepackage[empty]{fullpage}
\usepackage{titlesec}
\usepackage{marvosym}
\usepackage[usenames,dvipsnames]{color}
\usepackage{verbatim}
\usepackage{enumitem}
\usepackage[hidelinks]{hyperref}
\usepackage{fancyhdr}
\usepackage[english]{babel}
\usepackage{tabularx}
\usepackage{fontawesome5}
\input{glyphtounicode}
\usepackage{tikz}
\usepackage{multicol}

\newcommand{\ExternalLink}{%
    \tikz[x=1.2ex, y=1.2ex, baseline=-0.05ex]{% 
        \begin{scope}[x=1ex, y=1ex]
            \clip (-0.1,-0.1) 
                --++ (-0, 1.2) 
                --++ (0.6, 0) 
                --++ (0, -0.6) 
                --++ (0.6, 0) 
                --++ (0, -1);
            \path[draw, 
                line width = 0.5, 
                rounded corners=0.5] 
                (0,0) rectangle (1,1);
        \end{scope}
        \path[draw, line width = 0.5] (0.5, 0.5) 
            -- (1, 1);
        \path[draw, line width = 0.5] (0.6, 1) 
            -- (1, 1) -- (1, 0.6);
        }
}


%----------FONT OPTIONS----------
% sans-serif
% \usepackage[sfdefault]{FiraSans}
% \usepackage[sfdefault]{roboto}
% \usepackage[sfdefault]{noto-sans}
% \usepackage[default]{sourcesanspro}

% serif
% \usepackage{CormorantGaramond}
% \usepackage{charter}


\pagestyle{fancy}
\fancyhf{} % clear all header and footer fields
\fancyfoot{}
\renewcommand{\headrulewidth}{0pt}
\renewcommand{\footrulewidth}{0pt}

% Adjust margins
\addtolength{\oddsidemargin}{-0.5in}
\addtolength{\evensidemargin}{-0.5in}
\addtolength{\textwidth}{1in}
\addtolength{\topmargin}{-.5in}
\addtolength{\textheight}{1.0in}

\urlstyle{same}

\raggedbottom
\raggedright
\setlength{\tabcolsep}{0in}

% Sections formatting
\titleformat{\section}{
  \vspace{-5pt}\scshape\raggedright\large
}{}{0em}{}[\color{black}\titlerule \vspace{-5pt}]

% Ensure that generate pdf is machine readable/ATS parsable
\pdfgentounicode=1

%-------------------------
% Custom commands
\newcommand{\resumeItem}[1]{
  \item\small{
    {#1 \vspace{-2pt}}
  }
}

\newcommand{\resumeSubheading}[4]{
  \vspace{-2pt}\item
    \begin{tabular*}{0.97\textwidth}[t]{l@{\extracolsep{\fill}}r}
      \textbf{#1} & #2 \\
      \textit{\small#3} & \textit{\small #4} \\
    \end{tabular*}\vspace{-7pt}
}

\newcommand{\resumeSubSubheading}[2]{
    \item
    \begin{tabular*}{0.97\textwidth}{l@{\extracolsep{\fill}}r}
      \textit{\small#1} & \textit{\small #2} \\
    \end{tabular*}\vspace{-7pt}
}

\newcommand{\resumeProjectHeading}[2]{
    \item
    \begin{tabular*}{0.97\textwidth}{l@{\extracolsep{\fill}}r}
      \small#1 & #2 \\
    \end{tabular*}\vspace{-7pt}
}

\newcommand{\resumeSubItem}[1]{\resumeItem{#1}\vspace{-4pt}}

\renewcommand\labelitemii{$\vcenter{\hbox{\tiny$\bullet$}}$}

\newcommand{\resumeSubHeadingListStart}{\begin{itemize}[leftmargin=0.08in, label={}]}
\newcommand{\resumeSubHeadingListEnd}{\end{itemize}}
\newcommand{\resumeItemListStart}{\begin{itemize}[leftmargin=0.22in]}
\newcommand{\resumeItemListEnd}{\end{itemize}\vspace{-5pt}}

%-------------------------------------------
%%%%%%  RESUME STARTS HERE  %%%%%%%%%%%%%%%%%%%%%%%%%%%%


\begin{document}


\begin{center}
    \textbf{\Huge \scshape James Young} \\ \vspace{1pt}
    \small 
    \href{mailto:jamesyoung3931@gmail.com}{\faEnvelope { jamesyoung3931@gmail.com}} $|$ 
    \href{https://github.com/jamesyoung-15} {\faGithub\ {github.com/jamesyoung-15}} $|$
    \href{https://jyyoung.com}{\faGlobe { jyyoung.com}} % $|$
    % \href{https://linkedin.com/in/jamesyyoung}{\faLinkedin { linkedin.com/in/jamesyyoung}}
\end{center}


%-----------EDUCATION-----------
\section{Education}
  \resumeSubHeadingListStart
    \resumeSubheading
        {Boston University}{MA, United States}
        {Master of Science in Computer Science}{September 2024 -- Present}
        % \resumeItemListStart
            % \resumeItem{GPA: 4.0}
            % \resumeItem{Relevant Coursework: Analysis of Algorithms, Database Management, Operating Systems}
        % \resumeItemListEnd
    \resumeSubheading
        {Hong Kong University of Science and Technology}{Hong Kong}
        {Bachelor of Science in Electronic Engineering, Minor in Information Technology}{September 2020 -- June 2024}
        % \resumeItemListStart
            % \resumeItem{GPA: 3.13}
            % \resumeItem{Relevant Coursework: Cloud Computing and Big Data Systems, Deep Learning in Computer Vision}
        % \resumeItemListEnd
  \resumeSubHeadingListEnd

%-----------PROGRAMMING SKILLS-----------
\section{Skills}
 \begin{itemize}[leftmargin=0.08in, label={}]
    \small{\item{
     \textbf{Technologies}{: Python, Typescript, SQL (Postgres), MongoDB, HTML/CSS} \\
     \textbf{Libraries/Frameworks}{: React, FastAPI, Flask } \\
     \textbf{Platform/Tools}{: Git, AWS, Terraform, Linux, Docker} \\
    }}
 \end{itemize}


%-----------EXPERIENCE-----------
\section{Experience}
  \resumeSubHeadingListStart

  \resumeSubheading
    {Software Engineering Intern}{June 2025 -- December 2025}
    {Granite Telecommunications}{United States}
    \resumeItemListStart
      \resumeItem{Contributed automation scripts for Linux edge device system and network health monitoring, enabling automatic validation when new images are flashed to reduce manual testing overhead}
      \resumeItem{Worked on extracting and analyzing NOC ConnectWise ticket data, implementing text embedding and clustering techniques to identify patterns in customer support interactions}
      \resumeItem{Assisted in developing React and FastAPI internal chat application to interface with company's RAG-based LLM system trained on internal documents}
    \resumeItemListEnd

    \resumeSubheading
      {Software Developer Intern}{December 2023 -- February 2024}
      {Intelligent Design Technology Limited}{Hong Kong}
      \resumeItemListStart
        \resumeItem{Developed a prototype for real-time human fall detection for a Raspberry PI based robot in Python}
        \resumeItem{Utilized OpenCV for video capture and Tensorflow with Movenet for pose estimation combined with heuristics for classifying fall}
      \resumeItemListEnd
      
% -----------Multiple Positions Heading-----------
%    \resumeSubSubheading
%     {Software Engineer I}{Oct 2014 - Sep 2016}
%     \resumeItemListStart
%        \resumeItem{Apache Beam}
%          {Apache Beam is a unified model for defining both batch and streaming data-parallel processing pipelines}
%     \resumeItemListEnd
%    \resumeSubHeadingListEnd
%-------------------------------------------

    % \resumeSubheading
    %   {IoT Intern}{December 2022 -- May 2023}
    %   {Spotless Tech Limited}{Hong Kong}
    %   \resumeItemListStart
    %     \resumeItem{Implemented C++ libraries for reading water sensors with ESP32 and sending sensor data to AWS with MQTT, achieved ESP32 20\% power consumption reduction by implementing light sleep intervals.}
    %     \resumeItem{Worked on implementing peer-to-peer communication for activating nearby water pumps with multiple ESP32 devices. Increased communication range and speed by more than 10\% by switching from BLE to ESP-Now.}
    % \resumeItemListEnd

  \resumeSubHeadingListEnd

%-----------PROJECTS-----------
\section{Projects}
    \resumeSubHeadingListStart
        % Project 1
        \resumeSubheading
        {\textbf{{Bluebike Availability Predictor}}}{\emph{\href{https://bluebikespredictor.jyylab.com}{bluebikespredictor.jyylab.com}}}
        {Python, Flask, EC2, Terraform}{}
        \resumeItemListStart
          \resumeItem{Developed an application that makes real-time future predictions on the number of available Bluebikes at 3 bike stations near my home using machine learning.}
          \resumeItem{Employed Random Forest for prediction model trained with monthly Bluebike historical ridership data alongside temperature and precipitation data.}
          % \resumeItem{Integrated data pipeline with Lambda, S3, Athena, and EventBridge to pull Bluebike and weather data, store data in S3, and update the model monthly.}
          % \resumeItem{Created a PowerBI dashboard to visualize the predictions and historical data.}
          \resumeItem{Added web interface using Flask to display predictions and data visualizations, deployed with Docker on EC2 with Terraform.}
        \resumeItemListEnd

        % Project 2
        \resumeSubheading
            {\textbf{\href{https://faceblur.jyylab.com}{Serverless Face Blurrer}}}{\emph{\href{https://faceblur.jyylab.com}{faceblur.jyylab.com}}}
            {React, Python, Lambda, DynamoDB, S3, Rekognition, Terraform}{}
            \resumeItemListStart
                \resumeItem{Developed a full-stack application that automatically blurs faces in images uploaded to an S3 bucket with job queues stored in DynamoDB}
                \resumeItem{Frontend built with React and backend built with Python using AWS Lambda and Rekognition}
                \resumeItem{Deployed the application using Terraform with CI/CD implemented with Github Actions}
                \resumeItem{Improved system reliability by adding SQS to handle image processing jobs}
            \resumeItemListEnd

        % % Project 2
        % \resumeSubheading
        %     {\textbf{\href{https://github.com/jamesyoung-15/homelab}{Homelab}}}{\emph{\href{https://jyyhomelab.com}{jyyhomelab.com}}}
        %     {Linux, Docker, Ansible, Terraform}{}
        %     \resumeItemListStart
        %     \resumeItem{Built and managed a setup of servers and network equipment for experimenting, learning, and self-hosting services}
        %     \resumeItem{Utilized Proxmox for virtualization and OPNsense firewall for advanced network configurations (e.g., VLANs)}
        %     \resumeItem{Automated resource provisioning using Terraform and implemented system monitoring with Grafana}
        %     \resumeItemListEnd
    \resumeSubHeadingListEnd

% -----------Certifications-----------
\section{Certifications}
    \vspace{-15pt} % if using multicols
    % \vspace{1pt}
    \begin{multicols}{1}
        \begin{itemize}[itemsep=-1pt, parsep=3pt, leftmargin=0.22in]
        \small
            \item AWS Certified Solutions Architect (SAA-C03)
            \item Red Hat Certified System Administrator (RHCSA)
            \item HashiCorp Certified Terraform Associate (003)
            % \item CompTia A+ Certification
        \end{itemize}
    \end{multicols}
    % \begin{itemize}[itemsep=-1pt, parsep=3pt, leftmargin=0.22in]
        % \small{
            % \item AWS Certified Solutions Architect (SAA-C03)
            % \item Red Hat Certified System Administrator (RHCSA)
            % \item HashiCorp Certified Terraform Associate (003)
            % \item CompTia A+ Certification
        % }
    % \end{itemize}

% -----------Extracurriculars-----------
\section{Extracurriculars}
  \resumeSubHeadingListStart
    \vspace{-2pt}
    \resumeProjectHeading
      {\textbf{NCAE Cyber Games}}{February 2025 -- April 2025}
      \resumeItemListStart
        \resumeItem{Joined NCAE Cyber Games, a cybersecurity and CTF competition where our team competed against other universities and finished 1st in the regionals.}
      \resumeItemListEnd
      
  \resumeSubHeadingListEnd


%-------------------------------------------
\end{document}